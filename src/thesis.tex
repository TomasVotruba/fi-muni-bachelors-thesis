\documentclass[%
  %% This option enables the default options for the
  %% digital version of a document. Replace with `printed`
  %% to enable the default options for the printed version
  %% of a document.
  digital,
  %% This option enables colorful typesetting. Replace with
  %% `monochrome`, if you are going to print the thesis on
  %% a monochromatic printer.
  color,
  %% Causes the coloring of tables. Replace with `notable` to restore plain tables.
  table,
  %% Prints the List of Figures. Replace with `nolof` to hide the List of Figures.
  lof,
  %% Prints the List of Tables. Replace with `nolot` to hide the List of Tables.
  lot,
  %% More options are listed in the class documentation at
  %% <http://mirrors.ctan.org/macros/latex/contrib/fithesis/fithesis/guide/mu/fi.pdf>.
]{fithesis3}

%% The following section sets up the locales used in the thesis.
%% We need to load the T2A font encoding
\usepackage[resetfonts]{cmap}
%% to use the Cyrillic fonts with Russian texts.
\usepackage[T1,T2A]{fontenc}

%% By using `czech` or `slovak` as the main locale instead of `english`,
%% you can typeset the thesis in either Czech or Slovak, respectively.
%% The additional keys allow foreign texts to be typeset as follows:
\usepackage[%
  main=english,
  %% The additional keys allow foreign texts to be typeset as follows:
  czech, slovak
]{babel}
\usepackage{xcolor}
% \newcommand{\todo}[1]{\textcolor{red}{\textbf{#1}}}

\usepackage{listings}

%% The following section sets up the metadata of the thesis.
\thesissetup{%
    date          = \the\year/\the\month/\the\day,
    university    = mu,
    faculty       = fi,
    type          = bc,
    author        = Filip Procházka,
    gender        = m,
    advisor       = RNDr. Jaroslav Bayer,
    title         = {Refactoring of Kdyby packages},
    TeXtitle      = {Refactoring of Kdyby packages},
    keywords      = {package, kdyby, nette, doctrine, orm, composer, packagist},
    TeXkeywords   = {package, kdyby, nette, doctrine, orm, composer, packagist},
}
\thesislong{abstract}{%
    This is the abstract of my thesis, which can

    span multiple paragraphs.
}
\thesislong{thanks}{%
    This is the acknowledgement for my thesis, which can

    span multiple paragraphs.
}

%% When typesetting the bibliography, the `numeric` style will be used for the entries
%% and the `numeric-comp` style for the references to the entries.
%% The entries will be sorted in cite order.
%% For more unformation about the available `style`s and `citestyles`, see:
%% <http://mirrors.ctan.org/macros/latex/contrib/biblatex/doc/biblatex.pdf>.
\usepackage[utf8]{inputenc}
\usepackage{csquotes}
\usepackage[%
  style=numeric,
  citestyle=numeric-comp,
  sorting=none,
  sortlocale=auto
]{biblatex}

%% The bibliograpic database
\addbibresource{src/citations.bib}

%% The `makeidx` package contains helper commands for index typesetting.
\usepackage{makeidx}
\makeindex

%% These additional packages are used within the document:
\usepackage{paralist}
\usepackage{amsmath}
\usepackage{amsthm}
\usepackage{amsfonts}
\usepackage{url}
\usepackage{menukeys}
\usepackage{mwe}

\begin{document}

\chapter{Introduction}

The Kdyby is an Open-source software (OSS)~\cite{wiki:oss} project that I, Filip Procházka, lead and maintain. It is a set of PHP~\cite{wiki:php} libraries, that aim to ease writing of web applications.

Through my carer, I've been able to use the Kdyby in core business applications of companies such as Damejidlo.cz and Rohlik.cz. A lot of people consider my work useful enough, to incorporate it to their own applications as well.

As of writing this, the more popular libraries have hundreds of thousands of downloads. Five of Kdyby libraries have over quarter million downloads and one is approaching half a million with staggering amount of 470 thousands of downloads~\cite{packagist:kdyby}. In conclusion, a sober estimate would be, that Kdyby libraries are used in hundreds of real production applications.

If I'll account only for the two biggest projects that I can confirm are using the Kdyby packages, over a billion Czech crowns~\cite{tyinternety:rohlik-billion} has literary flowed through the Kdyby. That is a big responsibility.

Over the years, I've had problems keeping up with the demand and the packages began to get obsolete. I wanna use this thesis as way to fix the situation.

I'm going to review the state of each library, decide it's future, which means I'll either deprecate it and provide the users a suggestion for a better alternative, or fix the problems and refactor the library.

\chapter{Background for understanding the Kdyby}

\section{A brief history of Kdyby} \label{sec:theory:kdyby-history}

At 2006, few months after I've started attending my high school, I've began learning PHP and a friend introduced me to a concept called Content Management System (CMS)~\cite{wiki:cms}. I've immediately started working on my own CMS, as a way to learn PHP and also as a way to make some money.

I've managed to make a working prototype, that was used in production on few websites. The oldest preserved version is archived at \url{https://github.com/fprochazka/kdyby-cms-old}. And as expected, it is full of security holes and badly written code.

Then the concept of Open-source software (OSS)~\cite{wiki:oss} was introduced to me and I've decided to start working on everything openly, under a free license~\cite{wiki:fsl}. Sadly, since then, no new working version of the Kdyby CMS was ever released, because I've rewritten it from scratch exactly 10 times.

Around year 2012, I've realized this is not a good way to continue and split all useful code to separate libraries, that could be used more or less independently and have their own release cycle.

\section{Technologies used}

\subsection{Git and Github} \label{sec:theory:git}

Git is a Version Control System, that is decentralized and very fast.~\cite{progit} Github is a collaboration platform for software development using Git.

Each project has a page on Github called a repository, that can be used to inspect the Git history, files and other metadata. On the project repository page, there are issues and pull requests. Pull requests are a way to ask the maintainer of the repository, to incorporate provided code patch to the repository. It can be a bugfix, or new feature.

There are tools around pull requests that allow collaboration, code review and discussion about the provided code, so that the maintainers can help the contributors to provide the best code possible.

Kdyby is hosted and developed on Github.

\subsection{Continuous integration} \label{sec:theory:ci}

Continuous integration (CI) is the practice of merging all developer working copies to a shared mainline several times a day, to prevent merging conflicts.~\cite{wiki:ci} But now-days, when somebody speaks about CI, they usually mean CI servers, that run prepared task on the provided code.

In practice, it means that as the developer is working on a feature or bugfix, they push the work in progress code into a repository, the code is then picked up by a CI server, that executes the tests, checks coding style and runs various other tasks, to verify that the code is not broken and works.

When the work is finished and all the task on CI server completed with success, the code can be probably safely integrated, provided that the tests for new or changed functionality were added.

Some popular CI services are \fnurl{Travis~CI}{https://travis-ci.org/}, \fnurl{CircleCI}{https://circleci.com/} and \fnurl{GitLab~CI}{https://about.gitlab.com/features/gitlab-ci-cd/}. The Kdyby is using the Travis~CI.

\subsection{Composer} \label{sec:theory:composer}

Composer is a tool for dependency management~\cite{wiki:package-manager} in PHP. It allows you to declare the libraries your project depends on and it will manage (install or update) them for you~\cite{composer:docs:intro}.

Packages are usually published using Github with metadata in a file named \lstinline{composer.json}, that is written in JSON~\cite{wiki:json} format.

Composer is decentralized, but has a single main metadata repository \fnurl{Packagist}{https://packagist.org/}. It stores and provides all the package metadata like available versions and where to download them.

All Kdyby libraries are published as Composer packages on Packagist and installing them using the Composer is the only officially supported installation method.

\subsection{OAuth 2} \label{sec:theory:oauth2}

OAuth is a protocol for authentication and authorization, that can be implemented into a web service. It allows for secure exchange of user information, allowing third party websites to implement a login and registration process, that simplifies these tasks for the user. Essentially allowing him or her to login or register to a services through the OAuth 2 provider with two clicks.

Kdyby provides packages for integrating Nette Framework with OAuth 2 providers, such as \fnurl{Facebook}{https://developers.facebook.com/docs/facebook-login/manually-build-a-login-flow}, \fnurl{Google}{https://developers.google.com/identity/protocols/OAuth2} and \fnurl{Github}{https://developer.github.com/v3/oauth/}.

\subsection{Nette Framework} \label{sec:theory:nette}

Nette Framework is an OSS framework for creating web applications in PHP~\cite{wiki:nette}.

\subsection{dibi} \label{sec:theory:dibi}

Dibi is a Database Abstraction Library for PHP. It supports a lot of significant databases: MySQL, PostgreSQL, SQLite, MS SQL, Oracle, Access and generic PDO and ODBC.~\cite{dibi:homepage}

\subsection{Doctrine 2 ORM} \label{sec:theory:doctrine}

Doctrine ORM is an Object-Relation Mapper (ORM)~\cite{wiki:orm}, which means it allows the programmer to create PHP classes called entities, that represent relational data in a database and are used to actually map the data from the database to the classes and back. In conclusion, it allows the programmer to write a fully Object-oriented (OOP)~\cite{wiki:oop} applications.

\subsection{Symfony Framework} \label{sec:theory:symfony}

Symfony is a PHP web application framework and a set of reusable PHP components/libraries, similar to Nette.~\cite{wiki:symfony}

\subsection{Monolog} \label{sec:theory:monolog}

Monolog is a logging library that sends your logs to files, sockets, inboxes, databases and various web services. This library implements the PSR-3~\cite{fig:psr} interface that you can type-hint against in your own libraries to keep a maximum of interoperability.~\cite{monolog:readme}

\subsection{RabbitMQ} \label{sec:theory:rabbitmq}

RabbitMQ is OSS message broker software (sometimes called message-oriented middleware) that implements the Advanced Message Queuing Protocol (AMQP). The RabbitMQ server is written in the Erlang programming language and is built on the Open Telecom Platform framework for clustering and failover.~\cite{wiki:rabbitmq}

\subsection{ElasticSearch} \label{sec:theory:elasticsearch}

Elasticsearch is a search engine based on Lucene. It provides a distributed, multitenant-capable full-text search engine with an HTTP web interface and schema-free JSON documents. Elasticsearch is developed in Java and is released as open source under the terms of the Apache License. It is the most popular enterprise search engine .~\cite{wiki:elasticsearch}

\subsection{Redis} \label{sec:theory:redis}

Redis is an in-memory database OSS project, that is networked, in-memory, and stores keys with optional durability.~\cite{wiki:redis}

\subsection{PhpStan} \label{sec:theory:phpstan}

PHPStan focuses on finding errors in your code without actually running it. It catches whole classes of bugs even before you write tests for the code.~\cite{github:phpstan}

\subsection{Nette\textbackslash{}Tester} \label{sec:theory:nette-tester}

Nette\textbackslash{}Tester is an unit testing~\cite{wiki:unit-testing} framework for the PHP.~\cite{tester:docs}

\section{Techniques and design patterns}

\subsection{Dependency Injection} \label{sec:theory:di}

Inversion of control is a design principle in which custom-written portions of a computer program receive the flow of control from a generic framework.

Dependency injection is a technique whereby one object supplies the dependencies of another object. Passing the service to the client, rather than allowing a client to build or find the service, is the fundamental requirement of the pattern.~\cite{fowler:di}

\subsection{Aspect Oriented Programming} \label{sec:theory:aop}

In computing, aspect-oriented programming (AOP) is a programming paradigm that aims to increase modularity by allowing the separation of cross-cutting concerns. It does so by adding additional behavior to existing code (an advice) without modifying the code itself, instead separately specifying which code is modified via a "pointcut" specification, such as "log all function calls when the function's name begins with 'set'". This allows behaviors that are not central to the business logic (such as logging) to be added to a program without cluttering the code core to the functionality.~\cite{wiki:aop}

\subsection{Event Dispatcher} \label{sec:theory:event-dispatcher}

The Event Dispatcher is a pattern for writing modular code. It allows to create extension points in the library or application, that another library or application can hook into and change or extend the behavior.

Typically, the extension points are called hooks or events, and the new functionality is provided with objects called listeners.

\chapter{Current state of the project}

To be able to lay out the roadmap, first we have to know the current state of each Kdyby package, the original purpose and the current requirements. We shall only review those packages, that actually made it to production and at least one usable version was released.

Few years back I was really eager to solve all the problems around developing web applications in PHP and I've created few GitHub repositories as a reminder for me, to start working on those problems. And I've actually started to work on some, for example DoctrineForms is one of them, but it was never "officially released". The rest I've not even started working on.

\section{State of the project}

Lorem ipsum.

\section{State of each package}

Let's review each relevant package separately.

\subsection{Doctrine}

Lorem ipsum dolor sit amet.

\subsection{Console}

Lorem ipsum.

\subsection{Events}

Lorem ipsum.

\subsection{Annotations}

Lorem ipsum.

\subsection{DoctrineCache}

Lorem ipsum.

\subsection{DoctrineMagicAccessors}

Lorem ipsum.

\subsection{DoctrineCollectionsReadonly}

Lorem ipsum.

\subsection{DoctrineCollectionsLazy}

Lorem ipsum.

\subsection{DoctrineDbalBatchImport}

Lorem ipsum.

\subsection{DoctrineForms}

Lorem ipsum.

\subsection{Curl}

Lorem ipsum.

\subsection{CurlCaBundle}

Lorem ipsum.

\subsection{Autowired}

Lorem ipsum.

\subsection{FormsReplicator}

Lorem ipsum.

\subsection{Translation}

Lorem ipsum.

\subsection{Validator}

Lorem ipsum.

\subsection{RabbitMq}

Lorem ipsum.

\subsection{Money}

Lorem ipsum.

\subsection{DoctrineMoney}

Lorem ipsum.

\subsection{Aop}

Lorem ipsum.

\subsection{Clock}

Lorem ipsum.

\subsection{Redis}

Lorem ipsum.

\subsection{ParseUseStatements}

Lorem ipsum.

\subsection{RedisActiveLock}

Lorem ipsum.

\subsection{TesterParallelStress}

Lorem ipsum.

\subsection{Monolog}

Lorem ipsum.

\subsection{ElasticSearch}

Lorem ipsum.

\subsection{DoctrineSearch}

Lorem ipsum.

\subsection{Geocoder}

Lorem ipsum.

\subsection{CsobPaygateNette}

Lorem ipsum.

\subsection{CsobPaymentGateway}

Lorem ipsum.

\subsection{Wkhtmltopdf}

Lorem ipsum.

\subsection{FakeSession}

Lorem ipsum.

\subsection{RequestStack}

Lorem ipsum.

\subsection{StrictObjects}

Lorem ipsum.

\subsection{Facebook}

Lorem ipsum.

\subsection{Google}

Lorem ipsum.

\subsection{Github}

Lorem ipsum.

\subsection{NettePhpServer}

Lorem ipsum.

\subsection{TesterExtras}

Lorem ipsum.

\subsection{HtmlValidatorPanel}

Lorem ipsum.

\subsection{BootstrapFormRenderer}

Lorem ipsum.

\subsection{PayPalExpress}

Lorem ipsum.

\subsection{PresentersLocator}

Lorem ipsum.

\subsection{SvgRenderer}

Lorem ipsum.

\subsection{QrEncode}

Lorem ipsum.
\chapter{Designing roadmap of refactoring}

In this chapter, I am going to lay out the plan for the refactoring itself and set some specific goals for each package and for the project itself.

\section{Common requirements}

Lorem ipsum.

\section{Roadmap for each package}

\tocless\subsection{Doctrine}

Lorem ipsum.

\tocless\subsection{Console}

Lorem ipsum.

\tocless\subsection{Events}

Lorem ipsum.

\tocless\subsection{Annotations}

Lorem ipsum.

\tocless\subsection{DoctrineCache}

Lorem ipsum.

\tocless\subsection{DoctrineMagicAccessors}

Lorem ipsum.

\tocless\subsection{DoctrineCollectionsReadonly}

Lorem ipsum.

\tocless\subsection{DoctrineCollectionsLazy}

Lorem ipsum.

\tocless\subsection{DoctrineDbalBatchImport}

Lorem ipsum.

\tocless\subsection{DoctrineForms}

Lorem ipsum.

\tocless\subsection{Autowired}

Lorem ipsum.

\tocless\subsection{FormsReplicator}

Lorem ipsum.

\tocless\subsection{Translation}

Lorem ipsum.

\tocless\subsection{Validator}

Lorem ipsum.

\tocless\subsection{RabbitMq}

Lorem ipsum.

\tocless\subsection{Money}

Lorem ipsum.

\tocless\subsection{DoctrineMoney}

Lorem ipsum.

\tocless\subsection{Aop}

Lorem ipsum.

\tocless\subsection{Clock}

Lorem ipsum.

\tocless\subsection{Redis}

Lorem ipsum.

\tocless\subsection{ParseUseStatements}

Lorem ipsum.

\tocless\subsection{RedisActiveLock}

Lorem ipsum.

\tocless\subsection{TesterParallelStress}

Lorem ipsum.

\tocless\subsection{Monolog}

Lorem ipsum.

\tocless\subsection{ElasticSearch}

Lorem ipsum.

\tocless\subsection{DoctrineSearch}

Lorem ipsum.

\tocless\subsection{Geocoder}

Lorem ipsum.

\tocless\subsection{CsobPaygateNette}

Lorem ipsum.

\tocless\subsection{CsobPaymentGateway}

Lorem ipsum.

\tocless\subsection{Wkhtmltopdf}

Lorem ipsum.

\tocless\subsection{FakeSession}

Lorem ipsum.

\tocless\subsection{RequestStack}

Lorem ipsum.

\tocless\subsection{StrictObjects}

Lorem ipsum.

\tocless\subsection{Facebook}

Lorem ipsum.

\tocless\subsection{Google}

Lorem ipsum.

\tocless\subsection{Github}

Lorem ipsum.

\tocless\subsection{NettePhpServer}

Lorem ipsum.

\tocless\subsection{TesterExtras}

Lorem ipsum.

\tocless\subsection{HtmlValidatorPanel}

Lorem ipsum.

\chapter{Implementation}

\section{Doctrine}

Lorem ipsum.

\section{Console}

Lorem ipsum.

\section{Events}

Lorem ipsum.

\section{Annotations}

Lorem ipsum.

\section{DoctrineCache}

Lorem ipsum.

\section{DoctrineMagicAccessors}

Lorem ipsum.

\section{DoctrineCollectionsReadonly}

Lorem ipsum.

\section{DoctrineCollectionsLazy}

Lorem ipsum.

\section{DoctrineDbalBatchImport}

Lorem ipsum.

\section{DoctrineForms}

Lorem ipsum.

\section{Autowired}

Lorem ipsum.

\section{FormsReplicator}

Lorem ipsum.

\section{Translation}

Lorem ipsum.

\section{Validator}

Lorem ipsum.

\section{RabbitMq}

Lorem ipsum.

\section{Money}

Lorem ipsum.

\section{DoctrineMoney}

Lorem ipsum.

\section{Aop}

Lorem ipsum.

\section{Clock}

Lorem ipsum.

\section{Redis}

Lorem ipsum.

\section{ParseUseStatements}

Lorem ipsum.

\section{RedisActiveLock}

Lorem ipsum.

\section{TesterParallelStress}

Lorem ipsum.

\section{Monolog}

Lorem ipsum.

\section{ElasticSearch}

Lorem ipsum.

\section{DoctrineSearch}

Lorem ipsum.

\section{Geocoder}

Lorem ipsum.

\section{CsobPaygateNette}

Lorem ipsum.

\section{CsobPaymentGateway}

Lorem ipsum.

\section{Wkhtmltopdf}

Lorem ipsum.

\section{FakeSession}

Lorem ipsum.

\section{RequestStack}

Lorem ipsum.

\section{StrictObjects}

Lorem ipsum.

\section{Facebook}

Lorem ipsum.

\section{Google}

Lorem ipsum.

\section{Github}

Lorem ipsum.

\section{NettePhpServer}

Lorem ipsum.

\section{TesterExtras}

Lorem ipsum.

\section{HtmlValidatorPanel}

Lorem ipsum.

\chapter{Conclusion}

We made it!

\printbibliography

\end{document}
