\chapter{Introduction}

The Kdyby is an Open-source software (OSS)~\cite{wiki:oss} project that I, Filip Procházka, lead and maintain. It is a set of PHP~\cite{wiki:php} libraries, that aim to ease writing of web applications.

Through my carer, I've been able to use the Kdyby in core business applications of companies such as Damejidlo.cz and Rohlik.cz. A lot of people consider my work useful enough, to incorporate it to their own applications as well.

As of writing this, the more popular libraries have hundreds of thousands of downloads. Five of Kdyby libraries have over quarter million downloads and one is approaching half a million with staggering amount of 470 thousands of downloads~\cite{packagist:kdyby}. In conclusion, a sober estimate would be, that Kdyby libraries are used in hundreds of real production applications.

If I'll account only for the two biggest projects that I can confirm are using the Kdyby packages, over a billion Czech crowns~\footnote{Rohlik.cz loni prodal zboží za miliardu, letos chce konečně zisk \\\url{http://tyinternety.cz/e-commerce/rohlik-cz-loni-dosahl-na-miliardovy-obrat-letos-chce-konecne-zisk/}} has literary flowed through the Kdyby. That is a big responsibility.

Over the years, I've had problems keeping up with the demand and the packages began to get obsolete. I wanna use this thesis as way to fix the situation.

I am going to review the state of each library and decide its future. Which means I'll either deprecate it and provide the users a suggestion for a better alternative, or fix the problems and refactor the library.
