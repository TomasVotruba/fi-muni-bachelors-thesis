\chapter{Theory}

\section{A brief history of Kdyby} \label{sec:theory:kdyby-history}

At 2006, few months after I've started attending my high school, I've began learning PHP and a friend introduced me to a concept called Content Management System (CMS)~\cite{wiki:cms}. I've immediately started working on my own CMS, as a way to learn PHP and also as a way to make some money.

I've managed to make a working prototype, that was used in production on few websites. The oldest preserved version is archived at \url{https://github.com/fprochazka/kdyby-cms-old}. And as expected, it is full of security holes and badly written code.

Then the concept of Open-source software (OSS)~\cite{wiki:oss} was introduced to me and I've decided to start working on everything openly, under a free license~\cite{wiki:fsl}. Sadly, since then, no new working version of the Kdyby CMS was ever released, because I've rewritten it from scratch exactly 10 times.

Around year 2012, I've realised this is not a good way to continue and split all useful code to separate libraries, that could be used more or less independently and could have their own release cycle.

\section{What is a PHP package} \label{sec:theory:what-is-a-package}

A PHP package is a PHP library, that is ideally and developed using some Version Control System (VCS)~\cite{wiki:vcs} and published on Packagist~\ref{sec:theory:packagist} using Composer~\ref{sec:theory:composer}.

\section{Technologies used}

\subsection{Composer} \label{sec:theory:composer}

Composer is a tool for dependency management~\cite{wiki:package-manager} in PHP. It allows you to declare the libraries your project depends on and it will manage (install/update) them for you~\cite{composer:docs:intro}.

Packages are usually published using Github~\ref{sec:theory:github} as a Git~\cite{wiki:git} repository with metadata in file named \lstinline{composer.json}, that is written in format JSON~\cite{wiki:json}.

\subsection{Nette Framework} \label{sec:theory:nette}

Nette Framework is an OSS framework for creating web applications in PHP~\cite{wiki:nette}.

\subsection{Doctrine 2 ORM} \label{sec:theory:doctrine}

Doctrine ORM is an Object-Relation Mapper (ORM)~\cite{wiki:orm}, which means it allows the programmer to create PHP classes called entities, that represent relational data in database and are used to actually map the data from the database to the classes and back. In conclusin, it allows the programmer to write a fully Object-oriented (OOP)~\cite{wiki:oop} applications.

\subsection{Symfony Framework} \label{sec:theory:symfony}

Symfony is a PHP web application framework and a set of reusable PHP components/libraries.~\cite{wiki:symfony}

\subsection{Monolog} \label{sec:theory:monolog}

Monolog is a logging library that sends your logs to files, sockets, inboxes, databases and various web services. This library implements the PSR-3~\cite{fig:psr} interface that you can type-hint against in your own libraries to keep a maximum of interoperability.~\cite{monolog:readme}

\subsection{RabbitMQ} \label{sec:theory:rabbitmq}

RabbitMQ is OSS message broker software (sometimes called message-oriented middleware) that implements the Advanced Message Queuing Protocol (AMQP). The RabbitMQ server is written in the Erlang programming language and is built on the Open Telecom Platform framework for clustering and failover.~\cite{wiki:rabbitmq}

\subsection{ElasticSearch} \label{sec:theory:elasticsearch}

Elasticsearch is a search engine based on Lucene. It provides a distributed, multitenant-capable full-text search engine with an HTTP web interface and schema-free JSON documents. Elasticsearch is developed in Java and is released as open source under the terms of the Apache License. Elasticsearch is the most popular enterprise search engine .~\cite{wiki:elasticsearch}

\subsection{Redis} \label{sec:theory:redis}

Redis is an in-memory database OSS project, that is networked, in-memory, and stores keys with optional durability.~\cite{wiki:redis}

\subsection{PhpStan} \label{sec:theory:phpstan}

PHPStan focuses on finding errors in your code without actually running it. It catches whole classes of bugs even before you write tests for the code.~\cite{github:phpstan}

\subsection{Nette\textbackslash{}Tester} \label{sec:theory:nette-tester}

Nette\textbackslash{}Tester is a unit testing~\cite{wiki:unit-testing} framework for the PHP.~\cite{tester:docs}

\section{Techniques and design patterns}

\subsection{Dependency Injection} \label{sec:theory:di}

Inversion of control is a design principle in which custom-written portions of a computer program receive the flow of control from a generic framework.

Dependency injection is a technique whereby one object supplies the dependencies of another object. Passing the service to the client, rather than allowing a client to build or find the service, is the fundamental requirement of the pattern.~\cite{fowler:di}

\subsection{Aspect Oriented Programming} \label{sec:theory:aop}

In computing, aspect-oriented programming (AOP) is a programming paradigm that aims to increase modularity by allowing the separation of cross-cutting concerns. It does so by adding additional behavior to existing code (an advice) without modifying the code itself, instead separately specifying which code is modified via a "pointcut" specification, such as "log all function calls when the function's name begins with 'set'". This allows behaviors that are not central to the business logic (such as logging) to be added to a program without cluttering the code core to the functionality.~\cite{wiki:aop}

\section{3rd party services}

\subsection{Packagist} \label{sec:theory:packagist}

Packagist is the main Composer repository. It aggregates public PHP packages installable with Composer.~\cite{packagist:homepage}

\subsection{Facebook} \label{sec:theory:facebook}

Facebook is a social network, that can be used as an authentication tool for web services. If you have an account at Facebook and are logged in, some 3rd party service you'd like to use can use that, to allow you to log in to their website, without forcing you to go through registration process, or having to remember password.

\subsection{Github} \label{sec:theory:github}

Github is a colaboration platform for software development using Git~\cite{wiki:git}. Kdyby is hosted and developed here, such as many other Composer packages.

It can also, same as Facebook, be used as authentication tool.

\subsection{Google} \label{sec:theory:google}

Google is a company behind the most popular search engine google.com, but we will talk only about it's OAuth 2 functionality.

\subsection{PayPal} \label{sec:theory:paypal}

PayPal is a payment service, that you can use to pay on 3rd party websites, using your debit or credit card with very simple process.

\subsection{CSOB Payment Gateway} \label{sec:theory:csob-payment}

CSOB Payment Gateway is a standard card payment gateway provider.
